\section{Conclusão}


\begin{frame}{Conclusão}
  A LPA$E\tau$ foi capaz de controlar o sistema proposto para as configurações liga/desliga e PI.
  Os resultados são iniciais do ponto de vista da
  utilização no controle dinâmico de sistemas e apresenta-se como
  promissora a sua exploração associado com 
  inteligência artificial ou sistema adaptativo,
  para a geração dos parâmetros de controle.
\end{frame}






\begin{frame}{Conclusão}

Contribuições do trabalho
  
\begin{itemize}
\item Aplicação da LPA$E\tau$ em um sistema de controle;
\item Compreensão da LPA$E\tau$ e suas formas de aplicação;
\item Aplicação bem sucedida mediante requisitos de desempenho do sistema;
\item Apresentação de uma nova proposta para realização do controle dinâmico de sistemas;
\item Aplicação de um método de validação da nova proposta;
\item Investigação das possibilidades e áreas distintas de aplicação;
\item Ampliação do conhecimento sobre a LPA$E\tau$ sob uma perspectiva até então não explorada;
\item Possibilitar uma linha de pesquisa tendo como base o estudo da LPA$E\tau$ aplicada ao controle de sistemas;
\item Evidenciar possibilidades de trabalhos futuros;
  
\end{itemize}

\end{frame}


\begin{frame}{Conclusão}

Sugestões para trabalhos futuros:
  
\begin{itemize}
\item Controle de sistemas não lineares;
\item Aplicar o controlador LPA$E\tau$ em um sistema de segunda ordem e avaliar as implicações, limitações e potenciais;
\item Controle de sistemas críticos;
\item Utilizar um sistema operacional de tempo real para geranciar o controlador;
\item Melhoria da geração do parâmetro $\delta$, utilizando algum algoritmo adaptativo, inteligência alrificial ou alguma técnica que permita um melhor ajuste deste valor de correção.
\end{itemize}

\end{frame}
