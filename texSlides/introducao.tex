\section{Introdução}
\begin{frame}{Introdução}

%\centering
%\begin{minipage}{0.8\textwidth}

\begin{block}{Lógica Paraconsistente \tiny \cite{JoaoInacio}}
\begin{itemize}
\item Ferramenta promissora para tomada de decisão;
\item Robótica, Eng. Produção, Logística, Medicina, Automação e Controle, etc.
\end{itemize}
\end{block}

\begin{alertblock}{Ideia de uso da Lógica Paraconsistente \tiny \cite{JISF2011}}
\begin{itemize}
\item Conjunto de axiomas e regras de inferência;
\item Objetiva representar formalmente um raciocínio válido.
\end{itemize}
\end{alertblock}

%\end{minipage}
%\end{frame}


\end{frame}


%%%%%%%%%%%%%%%%%%%%%%%%%%%%%%%%%%%%%%% Objetivo
\begin{frame}{Objetivo(s)}
\begin{exampleblock}{Geral}
Realizar a análise e implementação da \textbf{ Lógica Paraconsistente Anotada Evidencial $E\tau$ (LPA$E\tau$)} em um sistema embarcado para atuar no controle dinâmico de um sistema físico.
\end{exampleblock}

\begin{alertblock}{Específicos}
  \begin{itemize}
    \item Estudar a LPA$E\tau$ e desenvolver um algoritmo a ser embarcado para atuar no controle de um sistema físico;
    \item Realizar a construção de um sistema físico para o controle de velocidade em um motor CC.
  \end{itemize}
\end{alertblock}

\end{frame}



%%%%%%%%%%%%%%%%%%%%%%%%%%%%%%%%%%%%%%% Relevância do trabalho
\begin{frame}{Relevância do Trabalho}
\begin{alertblock}{ }
  \begin{itemize}
    \item Iniciar pesquisa de aplicação da LPA$E\tau$ em sistemas de controle;
  \end{itemize}
\end{alertblock}
\begin{alertblock}{ }
  \begin{itemize}
    \item Balisar um novo caminho para trabalhos futuros, expondo pontos positivos, dificuldades iniciais e possibilidades para se trabalhar com a LPA$E\tau$ em sistemas de controle.
  \end{itemize}
\end{alertblock}
\end{frame}
