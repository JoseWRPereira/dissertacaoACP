%CONTEXTO: Identifica a grande area de pesquisa e sua importancia;

%LACUNA: (however:contudo, all do:todos fazem) O que precisa ser estudado nesse campo e ainda precisa ser entendido, exclarecido. Local aonde o trabalho está inserido;

%PRÓPOSITO: (this paper describes...)o que foi feito e principal objetivo do artigo;

%METODOLOGIA: Maneira geral falar dos métodos

%RESULTADOS: IMPORTANTÍSSIMO!!! Principal achado, resultado de forma bastante clara;

%CONCLUSOES: Mostrar como que os resutados contribuem para o avanço da grande aréa.

% Propósito
O objetivo deste estudo é mostrar uma implementação não convencional
de um controle em malha fechada utilizando a Lógica Paraconsistente
Anotada Evidencial $E\tau$ (LPA$E\tau$),
de forma a atender requisitos específicos de desempenho em um sistema físico proposto.
Sistemas de Controle são amplamente utilizados no setor industrial e
buscam uma maior eficiência de tempo e energia, mantendo uma qualidade
dos processos e do sistema controlado.
% Lacuna
O desenvolvimento de técnicas classificadas como Inteligência
Artificial fez surgirem outras opções para o controle de sistemas,
contudo ainda há escassez de implementações e testes usando técnicas alternativas.
% Metodologia
Para tanto é realizado um levantamento do modelo matemático desse
sistema físico, que é utilizado como referência para a implementação de um
controle clássico, com um controlador PI, bem como a definição dos
requisitos de desempenho do sistema, que norteiam a implementação
proposta utilizando a LPA$E\tau$, possibilitando assim a sua validação
por comparação entre os resultados dos controladores. 
% Resultados
Para o sistema testado,
os resultados entre os controladores
PI e LPA$E\tau$ foram equivalentes,
com ligeira vantagem a depender do
requisito de desempenho utilizado como referência.
Os resultados ainda mostraram algumas
possibilidades para futuras melhorias,
que se direcionam para plantas mais complexas,
como sistemas de segunda ordem,
e ainda permitem trabalhar com sistemas
sem a necessidade de conhecê-lo para controlá-lo,
adaptando seus parâmetros à ocasião
após algumas rodadas de aprendizado.
%Conclusão		
Os estudos e os resultados mostram um grande potencial para a
implementação e exploração da LPA$E\tau$ em sistemas de controle, de forma
semelhante as técnicas mais difundidas como uma Lógica Fuzzy, Redes
Neurais, Controle Adaptativo, Algorítmo Evolutivo, Inteligência
Artificial e Aprendizagem de máquina, inclusive
utilizando-as como suporte de geração de parâmetros de controle.


